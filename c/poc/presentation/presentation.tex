\documentclass{beamer}
\usepackage[utf8]{inputenc}

\usepackage{listings}

%% https://github.com/nickgian/thesis/blob/master/lstcoq.sty
\usepackage{color}

\definecolor{ltblue}{rgb}{0,0.4,0.4}
\definecolor{dkblue}{rgb}{0,0.1,0.6}
\definecolor{dkgreen}{rgb}{0,0.35,0}
\definecolor{dkviolet}{rgb}{0.3,0,0.5}
\definecolor{dkred}{rgb}{0.5,0,0}

% lstlisting coq style (inspired from a file of Assia Mahboubi)
%
\lstdefinelanguage{Coq}{ 
%
% Anything betweeen $ becomes LaTeX math mode
mathescape=true,
%
% Comments may or not include Latex commands
texcl=false, 
%
% Vernacular commands
morekeywords=[1]{Section, Module, End, Require, Import, Export,
  Variable, Variables, Parameter, Parameters, Axiom, Hypothesis,
  Hypotheses, Notation, Local, Tactic, Reserved, Scope, Open, Close,
  Bind, Delimit, Definition, Let, Ltac, Fixpoint, CoFixpoint, Add,
  Morphism, Relation, Implicit, Arguments, Unset, Contextual,
  Strict, Prenex, Implicits, Inductive, CoInductive, Record,
  Structure, Canonical, Coercion, Context, Class, Global, Instance,
  Program, Infix, Theorem, Lemma, Corollary, Proposition, Fact,
  Remark, Example, Proof, Goal, Save, Qed, Defined, Hint, Resolve,
  Rewrite, View, Search, Show, Print, Printing, All, Eval, Check,
  Projections, inside, outside, Def},
%
% Gallina
morekeywords=[2]{forall, exists, exists2, fun, fix, cofix, struct,
  match, with, end, as, in, return, let, if, is, then, else, for, of,
  nosimpl, when},
%
% Sorts
morekeywords=[3]{Type, Prop, Set, true, false, option},
%
% Various tactics, some are std Coq subsumed by ssr, for the manual purpose
morekeywords=[4]{pose, set, move, case, elim, apply, clear, hnf,
  intro, intros, generalize, rename, pattern, after, destruct,
  induction, using, refine, inversion, injection, rewrite, setoid_rewrite, congr,
  unlock, compute, ring, field, fourier, replace, setoid_replace, fold, unfold,
  change, cutrewrite, simpl, have, suff, wlog, suffices, without,
  loss, nat_norm, assert, cut, trivial, revert, bool_congr, nat_congr,
  symmetry, transitivity, auto, split, left, right, autorewrite},
%
% Terminators
morekeywords=[5]{by, done, exact, reflexivity, tauto, romega, omega,
  assumption, solve, contradiction, discriminate},
%
% Control
morekeywords=[6]{do, last, first, try, idtac, repeat},
%
% Comments delimiters, we do turn this off for the manual
morecomment=[s]{(*}{*)},
%
% Spaces are not displayed as a special character
showstringspaces=false,
%
% String delimiters
morestring=[b]",
morestring=[d]’,
%
% Size of tabulations
tabsize=3,
%
% Enables ASCII chars 128 to 255
extendedchars=false,
%
% Case sensitivity
sensitive=true,
%
% Automatic breaking of long lines
breaklines=false,
%
% Default style fors listings
basicstyle=\small,
%
% Position of captions is bottom
captionpos=b,
%
% flexible columns
basewidth={2em, 0.5em},
columns=flexible,
%
% Style for (listings') identifiers
identifierstyle={\ttfamily\color{black}},
% Style for declaration keywords
keywordstyle=[1]{\ttfamily\bfseries\color{dkviolet}},
% Style for gallina keywords
keywordstyle=[2]{\ttfamily\bfseries\color{dkgreen}},
% Style for sorts keywords
keywordstyle=[3]{\ttfamily\bfseries\color{ltblue}},
% Style for tactics keywords
keywordstyle=[4]{\ttfamily\color{dkblue}},
% Style for terminators keywords
keywordstyle=[5]{\ttfamily\color{dkred}},
%Style for iterators
%keywordstyle=[6]{\ttfamily\color{dkpink}},
% Style for strings
stringstyle=\ttfamily,
% Style for comments
commentstyle={\ttfamily\itshape\color{dkgreen}},
%
%moredelim=**[is][\ttfamily\color{red}]{/&}{&/},
literate=
    {fun}{{\color{dkgreen}{$\lambda\;$}}}1
    {bool}{{$\mathbb{B}$}}1
    {nat}{{$\mathbb{N}$}}1
    {Vforall2}{Vforall2}1 % quick workardoun to avoid partial replacement of 'forall' in identifier
    {nat\_equiv}{nat\_equiv}1 % quick workardoun to avoid partial replacement of 'nat' in identifier
    {forall}{{\color{dkgreen}{$\forall\;$}}}1
    {exists}{{$\exists\;$}}1
    {<-}{{$\leftarrow\;\;$}}1
    {=>}{{$\Rightarrow\;\;$}}1
    {==}{{\texttt{==}\;}}1
    {==>}{{$\Longrightarrow\;\;$}}1
%    {:>}{{\texttt{:>}\;}}1
    {->}{{$\rightarrow\;\;$}}1
    {<-->}{{$\longleftrightarrow\;\;$}}1
    {<->}{{$\leftrightarrow\;\;$}}1
    {<==}{{$\leq\;\;$}}1
    {\#}{{$^\star$}}1 
    {\\o}{{$\circ\;$}}1 
%    {\@}{{$\cdot$}}1 
    {\/\\}{{$\wedge\;$}}1
    {\\\/}{{$\vee\;$}}1
    {++}{{\texttt{++}}}1
    {~}{{\ }}1
    {¬}{{$\lnot$}}1     % this does not work
    {\@\@}{{$@$}}1
    {\\mapsto}{{$\mapsto\;$}}1
    {\\hline}{{\rule{\linewidth}{0.5pt}}}1
%
}[keywords,comments,strings]

\lstnewenvironment{coq}{\lstset{language=Coq}}{}

% pour inliner dans le texte
\def\coqe{\lstinline[language=Coq, basicstyle=\small]}
% pour inliner dans les tableaux / displaymath...
\def\coqes{\lstinline[language=Coq, basicstyle=\scriptsize]}

%%% Local Variables: 
%%% mode: latex
%%% Local IspellDict: british
%%% TeX-master: "main.tex"
%%% End: 

\title{Proving C program correct using C light operational semantics}
\date{ }

\begin{document}

\maketitle

\begin{frame}{Outline}
\begin{enumerate}
\item Formal verification - quick intro (high-level)
\item Coq mini intro 

\item Approach
\begin{itemize}   
\item Particular approach we consider: reasoning about C programs in Coq
\item Base PL concepts mini intro: syntax, AST, semantics.
\end{itemize}{}
\item Toy example: strlen Informal specification (man page)
\begin{itemize}

\item Formal specification of strlen (relational)
\item Simple implementation in C
\item From C program to AST using clightgen
\item Semantics of C program semantics and its equivalence to specification
\item Undefined behaviours in C and guarding against them
\end{itemize}{}
\item Conclusions
\end{enumerate} 
\end{frame}{}

% do the first two sections in the end, maybe ask boys
\section{Formal verification - quick intro}
\begin{frame}
CompCert example
\end{frame}
\begin{frame}{Coq intro}
\end{frame}

\section{Approach}

\begin{frame}
Explain what was done before: disadvantages and advantages of purely functional approach (Illya)
\end{frame}
\begin{frame}
What we try now and why.
\begin{itemize}
\item reason about the actual implementation
\item parse C code into an abstract syntax tree using C light generator of CompCert (not verified)
\item reason about the C light program using operational semantics
\end{itemize}
\end{frame}
\begin{frame}{C light syntax}
\begin{description}
\item[types]
\end{description}

\end{frame}



\begin{frame}{C light semantics}
Operational semantics: bigstep
\end{frame}


\section{Toy example: length of a C string}

\begin{frame}{Informal spec}

$\ldots$
DESCRIPTION        

 The strlen() function calculates the length of the string pointed to
 by s, excluding the terminating null byte.

 \bigskip

RETURN VALUE 

 The strlen() function returns the number of bytes in the string
 pointed to by s.

 \bigskip

CONFORMING TO 

 POSIX.1-2001, POSIX.1-2008, C89, C99, C11, SVr4, 4.3BSD.



\end{frame}

\begin{frame}
 To formalize the spec we need a formal model of C integers, pointers and memory model
\end{frame}



\begin{frame}[t,fragile]{Int and Pointer offset types}

Formalizations of machine integers modulo $2^N$ defined as a module type in CompCert \url{lib/Integers.v}.\\

\bigskip

A machine integer (type int) is represented as a Coq arbitrary-precision
integer (type Z) plus a proof that it is in the range 0 (included) to
modulus (excluded).

\bigskip

\begin{lstlisting}[language=Coq]
Record int: Type :=
mkint { intval: Z; intrange: -1 < intval < modulus }.
\end{lstlisting}

8, 32, 64-bit integers are supported, as well as 32 and 64-bit pointer offsets

\end{frame}

\begin{frame}{Memory model}
defined in CompCert \url{common/Memory.v}

\bigskip
a type [mem] of memory states, the following 4 basic
operations over memory states, and their properties:
\begin{itemize}
\item [load]: read a memory chunk at a given address;
\item [store]: store a memory chunk at a given address;
\item [alloc]: allocate a fresh memory block;
\item [free]: invalidate a memory block.
\end{itemize}
\end{frame}

\begin{frame}{Formal spec}
\lstinputlisting[language=Coq]{strlen_spec.v}     
\end{frame}

\begin{frame}{From C program to AST using clightgen}

\lstinputlisting[language=C]{strlen.c}

\end{frame}{}

\begin{frame}[fragile]{C light AST (loop of strlen)}

\begin{lstlisting}[language=Coq]
Definition f_strlen_loop := {|
fn_params := ((_s, (tptr tuchar)) :: nil);
fn_temps := ((_i, tuint) :: (_t1, (tptr tuchar)) :: (_t2, tuchar) :: nil);
fn_body := 
(Sloop
(Ssequence
(Ssequence
(Ssequence
  (Sset _t1 (Etempvar _s (tptr tuchar)))
  (Sset _s
    (Ebinop Oadd (Etempvar _t1 (tptr tuchar))
      (Econst_int (Int.repr 1) tint) (tptr tuchar))))
(Ssequence
  (Sset _t2 (Ederef (Etempvar _t1 (tptr tuchar)) tuchar))
  (Sifthenelse (Etempvar _t2 tuchar) Sskip Sbreak)))
(Sset _i
(Ebinop Oadd (Etempvar _i tuint) (Econst_int (Int.repr 1)
 tint)
  tuint)))
Sskip) |}.

\end{lstlisting}

\end{frame}

\lstlistoflistings

\end{document}